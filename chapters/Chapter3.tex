\chapter{Análise externa da organização}
\label{Chapter3} %for referencing in future use

\section{Análise PESTAL}

Análise PESTAL é uma ferramenta estratégica utilizada por empresas e organizações para avaliar o ambiente externo em que estão inseridas. Essa análise é feita considerando seis fatores: Políticos, Económicos, Sociais, Tecnológicos, Ambientais e Legais. O objetivo dessa análise é identificar as oportunidades e ameaças que o ambiente externo pode oferecer para a empresa ou organização, de modo que se possa desenvolver estratégias mais eficazes e se antecipar às mudanças do mercado. A análise PESTAL é uma ferramenta importante para o planeamento estratégico de uma empresa, pois permite uma visão ampla do ambiente em que ela está inserida, fornecendo informações valiosas para a tomada de decisões.\\


\noindent \textbf{Fatores políticos} %, fatores que incluem questões legais e regulamentares como por exemplo, Política Fiscal e Regulações Ambientais.
\begin{itemize}
    \item A Super Bock está sujeita a regulamentações governamentais que abrangem a fabricação e comercialização de bebidas alcoólicas em Portugal e nos mercados internacionais em que atua.
    \item Impostos elevados em bebidas alcoólicas afetam o custo dos produtos da Super Bock e reduzem a demanda do consumidor.
    \item Enfrentam pressão da opinião pública e de grupos de saúde para reduzir o consumo de álcool, o que pode afetar as vendas.
\end{itemize}




\noindent \textbf{Fatores Económicos} %, estes abordam o poder de compra da organização e o custo capital, para analisar temos que verificar que efeitos, por exemplo, de que forma as taxas de câmbio, inflação  juros afetam a empresa.
\begin{itemize}
    \item As crises económico podem afetar o poder de compra do consumidor e afetar negativamente as vendas da empresa.
    \item A Super Bock opera num mercado altamente competitivo, onde grandes empresas de bebidas alcoólicas competem por participação de mercado.
    \item A exportação para vários países e está sujeita a flutuações cambiais que podem afetar os seus lucros.
\end{itemize}

\noindent \textbf{Fatores Sociais} %, que incluem fatores culturais e distribuições geográficas da organização, inclui exemplos como estilos de vida, hábitos consumo e distribuição etária. \\
\begin{itemize}
    \item A empresa pode enfrentar mudanças nos hábitos de consumo dos consumidores, como a preferência por bebidas não alcoólicas ou por marcas mais saudáveis.
    \item A Super Bock tem a responsabilidade de adotar práticas sociais responsáveis em relação à promoção de álcool e ao impacto ambiental das suas atividades.
    \item O envelhecimento populacional pode afetar a demanda por bebidas alcoólicas.
\end{itemize}

\newpage
\noindent \textbf{Fatores Tecnológicos} %, o fator que aborda a terceirização da empresa. Avaliada através dos processos de investigação e automação da empresa. \\
\begin{itemize}
    \item A Super Bock deve estar atualizada com as últimas tendências tecnológicas, como a digitalização de processos e a introdução de novos produtos no mercado.
    \item A automação pode melhorar a eficiência e reduzir custos, mas também pode resultar em redução de empregos.
    \item O avanço tecnológico pode afetar o comportamento do consumidor em relação às preferências de compra e às expectativas de serviços e experiências.
\end{itemize}


\noindent \textbf{Fatores Ambientais} %, o fator que aborda a terceirização da empresa. Avaliada através dos processos de investigação e automação da empresa. \\
\begin{itemize}
    \item O cumprimento das regulamentações ambientais e adotar práticas sustentáveis em todas as suas atividades, incluindo a produção, embalagem e transporte.
    \item A mudança climática pode afetar a disponibilidade de matérias-primas e aumentar os custos de produção e transporte.
    \item A gestão de resíduos é um desafio para a indústria de bebidas, e a Super Bock deve implementar práticas de reciclagem e redução de resíduos para minimizar o impacto ambiental.
\end{itemize}

\noindent \textbf{Fatores Legais} %, o fator que aborda a terceirização da empresa. Avaliada através dos processos de investigação e automação da empresa. \\
\begin{itemize}
    \item A Super Bock é afetada por leis relacionadas a direitos autorais, marcas registadas e patentes.
    \item A empresa também é afetada por leis trabalhistas e fiscais, bem como leis de comércio internacional.
    \item Obrigatoriedade de seguir regulamentos rigorosos de segurança alimentar e de rotulagem de produtos.
\end{itemize}

\section{modelo de Hofstede}
\section{Estilos de Liderança}
\section{Análise SWOT}