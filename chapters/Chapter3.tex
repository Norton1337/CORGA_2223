\chapter{Análise externa da organização}
\label{Chapter3} %for referencing in future use

\section{Análise PESTAL}

Análise PESTAL é uma ferramenta estratégica utilizada por empresas e organizações para avaliar o ambiente externo em que estão inseridas. Essa análise é feita considerando seis fatores: Políticos, Económicos, Sociais, Tecnológicos, Ambientais e Legais. O objetivo dessa análise é identificar as oportunidades e ameaças que o ambiente externo pode oferecer para a empresa ou organização, de modo que se possa desenvolver estratégias mais eficazes e se antecipar às mudanças do mercado. A análise PESTAL é uma ferramenta importante para o planeamento estratégico de uma empresa, pois permite uma visão ampla do ambiente em que ela está inserida, fornecendo informações valiosas para a tomada de decisões.\\


\noindent \textbf{Fatores políticos} %, fatores que incluem questões legais e regulamentares como por exemplo, Política Fiscal e Regulações Ambientais.
\begin{itemize}
    \item A Super Bock está sujeita a regulamentações governamentais que abrangem a fabricação e comercialização de bebidas alcoólicas em Portugal e nos mercados internacionais em que atua.
    \item Impostos elevados em bebidas alcoólicas afetam o custo dos produtos da Super Bock e reduzem a demanda do consumidor.
    \item Enfrentam pressão da opinião pública e de grupos de saúde para reduzir o consumo de álcool, o que pode afetar as vendas.
\end{itemize}




\noindent \textbf{Fatores Económicos} %, estes abordam o poder de compra da organização e o custo capital, para analisar temos que verificar que efeitos, por exemplo, de que forma as taxas de câmbio, inflação  juros afetam a empresa.
\begin{itemize}
    \item As crises económico podem afetar o poder de compra do consumidor e afetar negativamente as vendas da empresa.
    \item A Super Bock opera num mercado altamente competitivo, onde grandes empresas de bebidas alcoólicas competem por participação de mercado.
    \item A exportação para vários países e está sujeita a flutuações cambiais que podem afetar os seus lucros.
\end{itemize}

\noindent \textbf{Fatores Sociais} %, que incluem fatores culturais e distribuições geográficas da organização, inclui exemplos como estilos de vida, hábitos consumo e distribuição etária. \\
\begin{itemize}
    \item A empresa pode enfrentar mudanças nos hábitos de consumo dos consumidores, como a preferência por bebidas não alcoólicas ou por marcas mais saudáveis.
    \item A Super Bock tem a responsabilidade de adotar práticas sociais responsáveis em relação à promoção de álcool e ao impacto ambiental das suas atividades.
    \item O envelhecimento populacional pode afetar a demanda por bebidas alcoólicas.
\end{itemize}

\newpage
\noindent \textbf{Fatores Tecnológicos} %, o fator que aborda a terceirização da empresa. Avaliada através dos processos de investigação e automação da empresa. \\
\begin{itemize}
    \item A Super Bock deve estar atualizada com as últimas tendências tecnológicas, como a digitalização de processos e a introdução de novos produtos no mercado.
    \item A automação pode melhorar a eficiência e reduzir custos, mas também pode resultar em redução de empregos.
    \item O avanço tecnológico pode afetar o comportamento do consumidor em relação às preferências de compra e às expectativas de serviços e experiências.
\end{itemize}


\noindent \textbf{Fatores Ambientais} %, o fator que aborda a terceirização da empresa. Avaliada através dos processos de investigação e automação da empresa. \\
\begin{itemize}
    \item O cumprimento das regulamentações ambientais e adotar práticas sustentáveis em todas as suas atividades, incluindo a produção, embalagem e transporte.
    \item A mudança climática pode afetar a disponibilidade de matérias-primas e aumentar os custos de produção e transporte.
    \item A gestão de resíduos é um desafio para a indústria de bebidas, e a Super Bock deve implementar práticas de reciclagem e redução de resíduos para minimizar o impacto ambiental.
\end{itemize}

\noindent \textbf{Fatores Legais} %, o fator que aborda a terceirização da empresa. Avaliada através dos processos de investigação e automação da empresa. \\
\begin{itemize}
    \item A Super Bock é afetada por leis relacionadas a direitos autorais, marcas registadas e patentes.
    \item A empresa também é afetada por leis trabalhistas e fiscais, bem como leis de comércio internacional.
    \item Obrigatoriedade de seguir regulamentos rigorosos de segurança alimentar e de rotulagem de produtos.
\end{itemize}

\section{Modelo de Hofstede}
O modelo de Hofstede é uma teoria que descreve as diferenças culturais entre países. Foi desenvolvido por Geert Hofstede, um investigador em gestão intercultural. Este modelo baseia-se em cinco dimensões culturais principais que afetam o comportamento das pessoas em diferentes culturas. Essas dimensões são:

\noindent \textbf{}
\begin{itemize}
    
\item Distância hierárquica (Power Distance): Refere-se à forma como as pessoas numa cultura lidam com a desigualdade social e de poder.
\item Individualismo vs. Coletivismo (Individualism vs. Collectivism): Refere-se à importância que uma cultura dá aos interesses individuais em relação aos interesses coletivos.

\item Masculinidade vs. Feminilidade (Masculinity vs. Femininity): Refere-se à importância que uma cultura dá a traços tradicionalmente considerados masculinos ou femininos.


\item Evitação da incerteza (Uncertainty Avoidance): Refere-se à tolerância de uma cultura à ambiguidade e à incerteza.


\item Orientação de longo prazo vs. Orientação de curto prazo (Long-Term vs. Short-Term Orientation): Refere-se à importância que uma cultura dá à tradição e à estabilidade em relação à inovação e à mudança.

\item Indulgência vs. Restrição (Indulgence vs. Restraint): Refere-se à extensão em que as culturas permitem a gratificação imediata de desejos e impulsos ou exigem a moderação e a supressão desses impulsos em nome de valores sociais mais elevados.

\end{itemize}

O modelo de Hofstede tem sido amplamente utilizado na pesquisa em gestão intercultural e tem ajudado as pessoas a compreender melhor as diferenças culturais em contexto global.


Dado este contexto teórico, este modelo aplica-se à SuperBock da seguinte forma:
\noindent \textbf{}
\begin{itemize}

\item Distância Hierárquica: Sendo que a Superbock é uma empresa com bastantes níveis de gestão, devido à sua dimensão, isto promove uma maior distânica hierárquica e pode influenciar o comportamento dos colaboradores.

\item Individualismo vs Coletivismo: A empresa preocupa-se em encontrar colaboradores com espírito de equipa para promover um ambiente coletivista. 

\item Masculinidade vs Feminilidade: Neste aspeto a SuperBock é equilibrada, visto que temos aspetos da masculinidade como a enfatização da competitividade e sucesso financeiro, mas também valorizam qualidade de produto e satisfação do cliente, aspetos mais associados à cultura feminina.

\item Evitação da Incerteza: Este aspeto é bastante comum entre as empresas portuguesas. A preocupação com previsibilidade e estabilidade dos colaboradores através de procedimentos e regras claras. Aspeto que se aplica também na organização em estudo.

\item Orientação Longo Prazo vs Orientação Curto Prazo: Um aspeto que na SuperBock é bastante equilibrado, pois acima de tudo querem manter a qualidade do produto que deu origem à organização como a conhecemos hoje, a cerveja, e aqui temos evidências de orientação a longo prazo. No entanto, a SuperBock também está sempre preocupada com o feedback dos clientes e às tendências geracionais e adaptam ou inovam produtos para atender às necessidades do mercado. As pessoas responsáveis por esta vertente na empresa, têm uma orientação mais vocacionada a curto prazo. 

\item Indulgência vs. Restrição: Podemos considerar que a empresa, sendo esta produtora de bebidas alcoólicas, está inserida numa cultura mais indulgente, que valoriza o prazer e a sociabilidade em torno do consumo de álcool. No entanto, promove moderação e consumo responsável, mostrando que também é importante a restrição.
\end{itemize}

\section{Análise SWOT}

A análise SWOT é uma ferramenta frequentemente usada pelas organizações como forma de observação crítica e analítica tanto das suas forças como fraquezas, oportunidades e ameaças. É um método simples, contudo, eficaz para uma rápida compreensão das características de um produto, serviço ou empresa. 

Permite uma avaliação fidedigna da posição de uma empresa no mercado e ajuda a planear estratégias de expansão, diversificação e investimento, garantido o sucesso das mesmas. Assim, antes de um projeto sair do papel, é possível antecipar quais são as chances de este prosperar e poder também calejar os seus pontos fracos. 

Porém, apesar do presumido uso no caso da Super Bock, não deixa de ser pertinente a sua tradição de investir e prosperar em ocasiões frequentemente inadequadas para este lançamento de novas ideias. 

Fazendo uma descrição mais técnica desta marca, podemos constatar o seguinte:\\

\noindent \textbf{Forças:}
\begin{itemize}
    
    \item Marca líder nacional; 

    \item Presença Internacional; 

    \item Variedade de produtos, como cerveja artesanal e sem álcool; 

    \item Métodos de produção e reaproveitamento de matérias-primas de vanguarda; 

Forte participação em iniciativas culturais e desportivas; 
\end{itemize}

\noindent \textbf{Fraquezas:} 
\begin{itemize}

\item Forte dependência do mercado português; 

\item A cerveja de qualidade continua a ser o seu produto de vanguarda, sem grande diversificação para outras categorias de bebidas; 
\end{itemize}

\noindent \textbf{Oportunidades:}

\begin{itemize}
\item Investir mais em outras bebidas que não a cerveja; 

\item Aumento da quota de mercado no estrangeiro; 

\item Inovações nos produtos dentro da categoria cervejeira; 

\item Investimento no mercado de bebidas não-alcoólicas e vegan; 
\end{itemize}

\noindent \textbf{Ameaças:}
\begin{itemize}
    
\item Mudanças nas tendências de consumo; 

\item Inflação das matérias-primas e meios de transporte das mesmas, dado que a sua fábrica continua localizada num polo industrial, longe dos seus fornecedores; 
\end{itemize}


