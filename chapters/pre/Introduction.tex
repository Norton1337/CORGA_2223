\chapter*{Introdução}

No âmbito da disciplina de Comportamento Organizacional (CORGA), será apresentado um estudo sobre a empresa Super Bock, uma das maiores cervejarias portuguesas. O objetivo deste estudo é analisar tanto os fatores internos quanto externos que afetam a empresa, a fim de entender o seu comportamento organizacional.



Para alcançar esse objetivo, serão utilizadas diversas ferramentas de análise, como a análise Pestal, que examina fatores políticos, econômicos, sociais, tecnológicos, ambientais e legais. Além disso, serão averiguados a missão, visão e valores da empresa, os estilos de liderança adotados pela organização e a análise SWOT, que examina as forças, fraquezas, oportunidades e ameaças enfrentadas pela empresa.



A organização do trabalho consistirá em contextualizar a empresa através da sua história, apresentando os valores, missão e visão da Super Bock. Em seguida, serão realizadas as análises externas e internas da organização, para entender melhor o seu comportamento organizacional e como ela se relaciona com o ambiente externo.



Este estudo será importante para entendermos como uma empresa tão relevante no mercado português consegue se manter competitiva e enfrentar os desafios do mercado. Com a análise Pestal, SWOT e outras ferramentas, será possível avaliar as forças e fraquezas da empresa e identificar oportunidades de melhoria para a sua gestão e desenvolvimento.