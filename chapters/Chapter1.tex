\chapter*{História da Super Bock}
\label{Chapter1} %for referencing in future use
\ref{Chapter1}

A história da Super Bock começa em 1927, quando a marca é registada em Portugal, no dia 3 de março, e despachada a 9 de novembro do mesmo ano. Em 1933, a cerveja expande-se para o mundo, tornando-se uma marca de referência no mercado internacional. Durante a Segunda Guerra Mundial, em 1942, as dificuldades de navegação no Atlântico levaram a Unicer, empresa detentora da marca, a utilizar apenas malte nacional, o que permitiu manter a produção da cerveja em Portugal. No ano seguinte, em 1943, a Super Bock já era uma das marcas de cerveja mais vendidas em Portugal.\\

Em 1944, a falta de lúpulo obrigou a reduzir o ritmo de produção e a estabelecer um regime de rateio na distribuição de cerveja entre os agentes. No entanto, a marca conseguiu manter-se no mercado e continuar a produzir a sua cerveja de qualidade. Em 1950, a Super Bock começou a patrocinar eventos desportivos, o que reforçou a sua presença junto do público português. Já em 1957, a marca inaugurou a sua cervejaria, projetada pelo arquiteto Arménio Losa, onde se iniciou a produção da cerveja. Dois anos depois, em 1959, foi inaugurada a fábrica da CUFP, na Rua de Júlio Dinis e Rua da Piedade, no Porto, onde se fabrica a Super Bock até hoje.\\

A partir de 1968, os camiões Super Bock tornaram-se o sistema de distribuição direta na cidade do Porto, um marco importante na história da marca. Em 1975, com a revolução do 25 de Abril, a Unicer foi nacionalizada, tornando-se uma empresa pública. A partir daí, a marca continuou a crescer e a consolidar a sua posição no mercado português, apresentando um novo rótulo em 1980. Em 1995, lançou o festival Super Bock Super Rock, um dos eventos musicais mais importantes em Portugal.\\

Em 2005, a Super Bock tornou-se a cerveja mais pedida em Portugal, consagrando-se como uma das marcas mais amadas pelos portugueses. Já em 2016, a marca inaugurou a Rádio SBSR.FM, uma plataforma online que promove a música e a cultura portuguesa. Hoje, a Super Bock é uma das marcas de cerveja mais importantes em Portugal e no mundo, com uma história rica e um legado que orgulha os portugueses.\\
