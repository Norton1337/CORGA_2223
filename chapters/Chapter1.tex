\chapter*{História da Super Bock}
\label{Chapter1} %for referencing in future use

A história da Super Bock começa em 1927, quando a marca é registada em Portugal, no dia 3 de março, e despachada a 9 de novembro do mesmo ano. Em 1933, a cerveja expande-se para o mundo, tornando-se uma marca de referência no mercado internacional. Durante a Segunda Guerra Mundial, em 1942, as dificuldades de navegação no Atlântico levaram a Unicer, empresa detentora da marca, a utilizar apenas malte nacional, o que permitiu manter a produção da cerveja em Portugal. No ano seguinte, em 1943, a Super Bock já era uma das marcas de cerveja mais vendidas em Portugal.\\

Em 1944, a falta de lúpulo obrigou a reduzir o ritmo de produção e a estabelecer um regime de rateio na distribuição de cerveja entre os agentes. No entanto, a marca conseguiu manter-se no mercado e continuar a produzir a sua cerveja de qualidade. Em 1950, a Super Bock começou a patrocinar eventos desportivos, o que reforçou a sua presença junto do público português. Já em 1957, a marca inaugurou a sua cervejaria, projetada pelo arquiteto Arménio Losa, onde se iniciou a produção da cerveja. Dois anos depois, em 1959, foi inaugurada a fábrica da CUFP, na Rua de Júlio Dinis e Rua da Piedade, no Porto, onde se fabrica a Super Bock até hoje.\\

A partir de 1968, os camiões Super Bock tornaram-se o sistema de distribuição direta na cidade do Porto, um marco importante na história da marca. Em 1975, com a revolução do 25 de Abril, a Unicer foi nacionalizada, tornando-se uma empresa pública. A partir daí, a marca continuou a crescer e a consolidar a sua posição no mercado português, apresentando um novo rótulo em 1980. Em 1995, lançou o festival Super Bock Super Rock, um dos eventos musicais mais importantes em Portugal.\\

Em 2005, a Super Bock tornou-se a cerveja mais pedida em Portugal, consagrando-se como uma das marcas mais amadas pelos portugueses. Já em 2016, a marca inaugurou a Rádio SBSR.FM, uma plataforma online que promove a música e a cultura portuguesa. Hoje, a Super Bock é uma das marcas de cerveja mais importantes em Portugal e no mundo, com uma história rica e um legado que orgulha os portugueses.\\


\chapter*{Missão da Super Bock}

A Super Bock, fundada em 1927, tem como principal objetivo, ser líder na indústria da cerveja portuguesa. E como é que isto é possível? Este feito é atingido através do cumprimento de alguns fatores chave, como indicado pelos responsáveis pela empresa.\\

De modo a ocuparem este lugar de vanguarda tão firmemente, é feita a disponibilização de produtos economicamente sustentáveis e de elevada qualidade, através do uso dos melhores ingredientes e as mais avançadas técnicas de produção, que permitem um reduzido consumo de água e energia nas suas instalações fabris, aumentando a eficiência da sua cadeia de produção e inversamente proporcional, o seu impacto no meio ambiente. Contudo, seria injusto resumir o sucesso deste grupo a uma componente técnica extremamente avançada.\\

Não só vistos como um benchmark em termos de bebidas, nomeadamente cerveja, o impacto positivo que possuem manifesta-se na sociedade e no reconhecimento/valorização pela comunidade onde se insere e atraindo acionistas que investem no sucesso desta empresa. Aliás, também fabricam impressionantes resultados financeiros que, consequentemente, impulsionam o desenvolvimento económico português e a difusão do nosso mercado para o resto do mundo. Afinal, em 2021 foram distinguidos enquanto vencedores portugueses no World Beer Awards, competição que premeia e promove os diferentes estilos de cerveja no mercado.\\ 

Esta constante evolução deve-se à paixão e à motivação para o desenvolvimento dos melhores produtos possíveis. Contudo, este sucesso portuense não seria possível sem uma devida cultura organizacional, que valoriza o empreendedorismo e a sugestão de novas ideias. Esta filosofia surgiu ao acreditarem que para não estagnar o seu progresso, devem apostar no crescimento, formação e valorização da workforce humana que representa a sua marca (e como mencionado, o seu melhor ativo). Constantemente surpreender até o cliente mais exigente e, claro, manterem-se fiéis aos seus produtos com mais êxito.\\
 

Para além de fortes apoiantes do fornecimento de excelentes condições aos seus trabalhadores, através de um ambiente inclusivo e diversificado, têm um historial de providenciarem excelentes benefícios. Como forma de integração em diferentes áreas da sociedade, participam também num largo número de iniciativas sob a forma de eventos sociais e desportivos, por exemplo.\\

Deste modo, são positivamente influentes entre o médio e longo prazo, o que torna esta marca icónica!\\ 
