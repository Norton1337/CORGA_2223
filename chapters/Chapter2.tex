\chapter{Análise interna da organização}
\label{Chapter2} %for referencing in future use

\section{Cultura Diagnóstico da empresa e incorporação de dimensões associadas à Cultura Organizacional}

A Super Bock é uma empresa portuguesa com uma cultura organizacional sólida e comprometida com a excelência. A empresa valoriza a diversidade na sua equipa, reconhecendo que a diversidade é uma vantagem competitiva que permite obter resultados mais inovadores e criativos. Por isso, a Super Bock tem o cuidado de analisar o perfil do colaborador para que este se sinta integrado e respeitado na equipa em que vai trabalhar.

A cultura organizacional da Super Bock é um dos fatores que a torna uma das empresas mais bem-sucedidas em Portugal. A empresa é comprometida com a educação e promove causas aplicáveis ao tema, como a formação dos seus colaboradores e a oferta de programas de estágio para jovens. Além disso, a Super Bock tem um forte compromisso com a sustentabilidade, a responsabilidade social e a ética nos negócios, o que se reflete nas suas práticas e políticas.

A Super Bock tende a recrutar pessoas mais jovens para que possam desenvolver-se dentro da empresa. A empresa valoriza a capacidade de adaptação dos colaboradores ao ambiente de trabalho, entendendo que esta é uma habilidade fundamental para o sucesso nos negócios. Assim, a empresa oferece oportunidades para que os colaboradores possam desenvolver novas competências e habilidades.

O perfil procurado pela Super Bock é de pessoas que sabem trabalhar em equipa, são responsáveis, corajosas, têm iniciativa e são capazes. A empresa procura pessoas que tenham paixão pelo que fazem e estejam dispostas a trabalhar arduamente para atingir os objetivos da empresa.

Por fim, a Super Bock está em constante mudança organizacional, buscando melhorias na sua cultura e políticas internas. A empresa está sempre preocupada com a educação e a seleção criteriosa dos seus colaboradores. Além disso, a Super Bock tem um forte compromisso com a inovação e a melhoria contínua dos seus produtos e processos, o que a torna uma das empresas mais respeitadas e admiradas em Portugal. 


\section{Gestão da Mudança}
Tal como a semântica aponta, a gestão da mudança é um conjunto de boas práticas, sequência de ações ou processos utilizados numa alteração de amplitude variável na infraestrutura de uma organização. Seja a sua estratégia, os seus processos, a sua tecnologia, a cultura ou a hierarquia. Esta é o meio para atingir um \textit{endgoal} projetado previamente, com um propósito de incrementar a qualidade do meio de trabalho ou do produto/serviço entregue ao cliente final. 

Analisando o mercado em que a Super Bock se insere, a gestão da mudança deve ser uma atividade praticamente contínua dado que as tendências de consumo variam frequentemente e os métodos de produção são constantemente atualizados. Especialmente agora, com o exponencial desenvolvimento e incorporação da inteligência artificial, são obrigados a que os desenvolvimentos tecnológicos ocorram mais frequentemente, impedindo que os concorrentes reduzam a sua distância a este grupo. 

Esta tendência de rápida transição é notória desde o começo da história da empresa. Por exemplo, quando a localização da fábrica foi transferida da Praça da Galiza para Leça do Balio, onde ainda se situa, foram os primeiros a atribuírem o benefício de pequeno-almoço gratuito para os seus trabalhadores, visto que vinham exaustos a pé, numa altura em que transportes públicos eram escassos, tal como viaturas privadas.  

Na época de adoção desta medida, em que os benefícios prestados pelas empresas eram reduzidos, fizeram furor ao oferecerem ao providenciarem um completo seguro de saúde aos seus colaboradores, por exemplo. Apesar de atualmente ser algo extremamente banal em múltiplas organizações, numa empresa com quase 100 anos de história, este foi um ponto marcante na tendência de valorização da \textit{task force} humana. Assim, com uma mudança da perceção do papel humano no desempenho da empresa, foi possível aumentar a qualidade de vida dos seus trabalhadores. 

Num programa de \textit{trainees}, são também extremamente inclusivos e abertos a acrescentar membros nas equipas que compõem esta marca, não receando o acolher de talento. Com uma componente mais técnica, são também perspicazes a efetuar a mudança, contudo, são cautelosos no estudo da arte das tecnologias com que vão atualizar a sua cadeia de produção. 

\section{Estilos de Liderança}

O sucesso da Super Bock é atribuído não apenas à sua expertise técnica, mas também à sua cultura organizacional, que encoraja o empreendedorismo e a sugestão de novas ideias. A filosofia da empresa enfatiza o crescimento, o treino e o desenvolvimento da sua força de trabalho para manter a sua liderança na indústria.

A empresa valoriza o desenvolvimento de produtos de alta qualidade utilizando avançadas técnicas de produção e os melhores ingredientes, minimizando o impacto ambiental. A Super Bock também valoriza a inclusão, a diversidade e a colaboração no local de trabalho, além de fornecer excelentes benefícios aos seus funcionários. Eles oferecem seguro de saúde e focam-se em manter um equilíbrio saudável entre a vida pessoal e a profissional. Os seus valores estão alinhados com a sustentabilidade e os Objetivos de Desenvolvimento Sustentável da ONU. Tudo isso mostra que o estilo de liderança pode ser descrito como inovador, sustentável e focado na criação de valor para a empresa, nos seus funcionários e nas comunidades em que está presente. Com tudo isto em conta, categorizamos a liderança da Super Bock como Transformacional.

Líderes transformacionais são tipicamente indivíduos enérgicos, entusiasmados e comprometidos em ajudar cada membro do grupo a ter sucesso. A sua liderança envolve trabalhar para além dos interesses imediatos, identificar mudanças necessárias, criar uma visão para guiar a mudança, inspirar e influenciar os membros a executar a mudança colaborativamente.


\section{Estratégias de Motivação}
