\chapter{Resposta ao desafio}
\label{Chapter4} %for referencing in future use
Decidimos responder ao desafio seguinte: “Considerando que o futuro recorrerá a valências cada vez mais digitais, e que palavras como upskilling e reskilling começam a ser lugares-comuns, como deve uma organização como a Super Bock Group preparar um programa abrangente de capacitação digital para os seus quadros?"\\

Ao analisar o papel cada vez mais vital das habilidades digitais no futuro, organizações como a Super Bock Group devem investir em programas abrangentes de upskilling e reskilling digitais para os seus funcionários. Ao fazê-lo, a organização permanecerá competitiva e adaptar-se-á ao cenário tecnológico em constante evolução. É de notar que upskilling refere-se ao ensino de novas habilidades relevantes para o trabalho atual ou progressão de carreira futura dos funcionários, enquanto upskilling envolve treinar os funcionários para novos papéis emergentes devido aos avanços tecnológicos.\\

Para desenhar um programa de formação digital eficaz, a Super Bock Group pode considerar a implementação de várias estratégias. Em primeiro lugar, garantir um programa de treino personalizado ao realizar uma avaliação das habilidades dos seus funcionários para entender as necessidades da força de trabalho. Em segundo lugar, a parceria com provedores externos de treino em habilidades digitais pode fornecer acesso aos materiais de treino mais recentes e à expertise no campo, permitindo que a organização acompanhe as últimas tendências do setor. Em terceiro lugar, oferecer aos funcionários oportunidades de aprender por meio de experiência prática pode ser altamente benéfico. As oportunidades podem envolver projetos, estágios ou rotações de emprego, permitindo que os funcionários desenvolvam suas habilidades em cenários do mundo real.\\

Por último, mas não menos importante, criar uma cultura de aprendizagem e desenvolvimento contínuos, promovendo os benefícios de upskilling e reskilling digital, fornecendo incentivos para que os funcionários participem no treino e reconhecendo e recompensando aqueles que concluírem os programas de treino. Ao implementar estas estratégias, a Super Bock Group irá preparar a sua força de trabalho para o futuro, enquanto aumenta o envolvimento e a retenção dos funcionários. Funcionários que se sentem valorizados e apoiados provavelmente permanecerão leais à sua organização e motivados a alcançar os seus objetivos pessoais e profissionais. Além disso, uma força de trabalho habilidosa e conhecedora tem produtividade e inovação aumentadas, impulsionando o sucesso dos negócios. Investir em programas de upskilling e reskilling digital é uma decisão sábia para qualquer organização que tem como objetivo prosperar na era digital.
